\documentclass[a4paper,12pt]{article}  % or "report" for larger documents

\usepackage{polski}  % Polish language support
\usepackage[utf8]{inputenc}  % UTF-8 encoding
\usepackage{amsmath, amssymb}  % Math support
\usepackage{float}
\usepackage{graphicx}  % Insert images
\usepackage{hyperref}  % Clickable links
\usepackage{biblatex}  % Bibliography management
\addbibresource{references.bib}  % Reference file
\renewcommand{\figurename}{rys.}

\title{Zadanie konkursowe, eliminajce:\\ Strefa zgniotu}
\author{
  Milosz, Tesiorowski\\
  \and
  Maksymilian Borowy\\
  \and
  Milosz Medwid\\
  \and
  Franciszek Fabinski\\
}
\date{\today}

\begin{document}

\maketitle  % Generates title page

% Wymagania i punktacja:
% 
% Zrozumienie działania strefy zgniotu – 30 pkt.
% Kreatywność i innowacyjność – 30 pkt.
% Estetyka dokumentacji – 20 pkt.
% 
% im fajniejsze beda foty tym lepiej wyjdziemy

\section{Wstep teoretyczny}
strefa zgniotu polega na rozłożeniu energii kinetycznej na duży obszar w
trakcie kolizji przez największą ilość czasu, przez co pasażerowie będą 
narażeni na o wiele mniejsze przeciążenie i co za tym idzie zmniejsza się szansa 
na odniesienie uszczerbku na zdrowiu. Aby strefa zgniotu działała prawidłowo należy 
dobrać materiały podatne na deformację takie jak plastiki oraz aluminium. W niej ważne 
również jest użycie belek, ram i podłużnic, aby skutecznie przekierować siłę z dala od kabiny pojazdu.

Ważnymi wzorami, które opisują wartości fizyczne dla strefy zgniotu są:
\begin{equation}
  t = 2 \cdot \frac{s}{v}
\end{equation}

Dzięki temu wzorowi możemy obliczyć czas, w którym następuje kolizja:

\begin{equation}
  a = \frac{v^2}{2 \cdot s}
\end{equation}

Dzięki temu wzorowi możliwe jest obliczenie przyspieszenia, a po przeliczeniu na ilość przyspieszeń ziemskich można obliczyć jak mocne przeciążenie powstanie.

\begin{equation}
  F_{\text{avg}} \delta t = m \delta v
\end{equation}

Tym wzorem obliczymy siłę nałożoną na pojazd rozłożoną w czasie.


\subsection{Zasady BHP}

Projekt realizuje zasady BHP:
Chroni użytkownika: przez zabezpieczenie akumulatora, silnika i elektroniki (ograniczenie ryzyka zwarcia, przegrzania, wybuchu), a także pojazd: dzięki sprężystym spiralom ograniczającym uszkodzenia, również chroni środowisko pracy modelu: zapobiega oderwaniu się elementów, które mogłyby stanowić zagrożenie np. dla innych modeli lub osób.

Spełnia założenia bezpiecznego użytkowania modeli RC:
brak ostrych krawędzi,
kontrolowane pochłanianie sił,
trwałość konstrukcji,
możliwość łatwej wymiany elementów po kolizji.

\section{Analiza dostepnych technologii}

Projekt wykorzystuje dostępne technologię, co czyni go nie tylko przystępnym dla użytkowników ale też ekologicznym i bezpiecznym.wykorzystuje technologie dostępne w modelarstwie RC:
sprężyny tłumiące/spiralne,
tworzywa sztuczne (ABS, PLA),
druk 3D jako metoda wykonania.

Odnosi się do rozwiązań z motoryzacji, np.:

elastyczne struktury deformujące
absorbery zderzeń 
Jest oparty o komponenty powszechnie dostępne w sklepach modelarskich lub do druku 3D – łatwe do wdrożenia i testowania.
Spełnia wymóg przystosowania do realnych możliwości technicznych

\printbibliography

\end{document}

