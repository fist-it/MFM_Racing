\documentclass[a4paper,12pt]{article}  % or "report" for larger documents

\usepackage{polski}  % Polish language support
\usepackage[utf8]{inputenc}  % UTF-8 encoding
\usepackage{amsmath, amssymb}  % Math support
\usepackage{float}
\usepackage{graphicx}  % Insert images
\usepackage{hyperref}  % Clickable links
\usepackage{biblatex}  % Bibliography management
\addbibresource{references.bib}  % Reference file
\renewcommand{\figurename}{rys.}

\title{Zadanie konkursowe, eliminajce:\\ Strefa zgniotu}
\author{
  Milosz, Tesiorowski\\
  \and
  Maksymilian Borowy\\
  \and
  Milosz Medwid\\
  \and
  Franciszek Fabinski\\
}
\date{\today}

\begin{document}

\maketitle  % Generates title page

% Wymagania i punktacja:
% 
% Zrozumienie działania strefy zgniotu – 30 pkt.
% Kreatywność i innowacyjność – 30 pkt.
% Estetyka dokumentacji – 20 pkt.
% 
% im fajniejsze beda foty tym lepiej wyjdziemy

\section{Wstep teoretyczny}
% tu bedzie "zrozumienie" dzialania strefy zgniotu

\begin{equation}
  % jakis wzor
\end{equation}

% cos do opisu wzoru

\subsection{Zasady BHP}
% cos o BHP bo kazali w celu zadania??

\section{Analiza porownawcza}
% Porownanie rozwiazan strefy zgniotu, jesli bedzie za malo
% dla samochodow RC to moze byc dla normalnych

\section{Autorska koncepcja}
\subsection{subsekcja}
\subsection{Inna subsekcja}

\printbibliography

\end{document}
